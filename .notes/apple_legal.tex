tex
\documentclass[11pt]{letter}
\usepackage[utf8]{inputenc}
\usepackage{geometry}
\geometry{margin=1in}
\usepackage{setspace}
\usepackage{parskip}

\begin{document}

    \begin{letter}{Apple Inc.\\
    c/o CT Corporation System\\
    Registered Agent for Service of Process\\
    [Insert DE or CA Address Here]}

        \opening{To Whom It May Concern:}

        \textbf{RE: Executive-Level Review Request – Platform Abandonment, Legal Exposure, and Strategic Remediation Proposal}

        I am writing to formally request an executive-level discussion under the governance of Apple Legal regarding four urgent and legally actionable matters stemming from Apple’s current platform practices. These include deceptive business conduct, legacy product abandonment, privacy violations, and compliance failures — each of which exposes Apple to material legal, reputational, and financial risk.

        While I am fully prepared to escalate these matters through litigation or regulatory channels, my preference is to resolve them collaboratively. This letter is being delivered via formal process service to ensure internal visibility, establish a clear paper trail, and convey the seriousness of the issues.

        \section*{I. Executive Summary of Causes of Action}

        \begin{enumerate}
            \item \textbf{Forced Billing Relationship for Free or Previously Purchased Apps} \\
            Apple continues to require users to verify or maintain billing information in order to download free applications or update previously purchased ones. This constitutes an illegal tying arrangement and a deceptive trade practice. There is no lawful justification to require billing credentials for \$0.00 transactions or for software the customer has already paid for. It further violates principles of informed consent under GDPR and CCPA.

            \item \textbf{Bait-and-Switch Abandonment of Apple Software Products} \\
            Apple has orphaned numerous software products (e.g., Apple Remote Desktop, Server.app, iBooks Author) without assigning support teams, offering migration paths, or publicly disclosing deprecation. This is particularly acute in Apple Remote Desktop, which continues to generate significant App Store revenue — estimated at \$70M annually — while receiving no meaningful updates or support. Apple profits from this software while leaving customers legally and operationally stranded.

            \item \textbf{Forced Reboots Without User Consent} \\
            Apple has repeatedly pushed forced reboots for updates and security patches, disrupting workflows without user approval or override. These actions violate user autonomy, enterprise SLA expectations, and may result in negligent data loss or downtime, particularly in unattended production environments.

            \item \textbf{Unauthorized Migration of User Files to iCloud} \\
            Apple's iCloud ``Desktop \& Documents'' feature silently moves local files into cloud-only locations, often rendering them unavailable offline. This is not transparent sync behavior — it is file relocation without informed consent. Users are often unaware that their files are no longer stored locally. This practice has led to significant user harm, data confusion, and forced iCloud storage upgrades — all without meaningful opt-in.
        \end{enumerate}

        \section*{II. Historical Context and Platform Trust}

        I am not a casual observer. I was the original author of \textbf{QuickTime for Windows} and the architect of \textbf{QTML (QuickTime Media Layer)} — the cross-platform runtime system that enabled iTunes for Windows, supported legacy Mac OS applications under XNU, and laid the technical groundwork for Carbon. These technologies underpinned Apple's expansion into the Windows ecosystem and directly contributed to the iPod’s success and Apple’s financial recovery.

        More importantly: I personally \textbf{discovered and instigated Apple’s legal action against Microsoft and Intel} in the 1998 IP case, worked alongside the internal legal team to coach depositions and hearings, and was invited to review Apple’s IP portfolio for additional infractions. I held trusted access at the highest levels of technical and legal engagement and supported Apple through one of the most consequential platform integrity battles in its history.

        I cite this not for vanity — but to emphasize that I understand Apple’s historical standards, legal posture, and developer responsibility. These current policies betray that legacy.

        \section*{III. Strategic Remediation Offer}

        Rather than pursue a public or adversarial route, I am offering Apple a structured, confidential alternative:

        \textbf{I propose the creation of a risk containment and platform recovery function}, governed by Apple Legal, and operated through a direct, scoped partnership with me.

        \textbf{I will:}
        \begin{itemize}
            \item Resume active engineering support for Apple-authored legacy software still in distribution
            \item Perform remediation and modernization where necessary (e.g., ARD, Server.app, Automator)
            \item Address legacy security exposures, including iTunes for Windows and associated QuickTime subsystems
            \item Provide long-term maintenance and trusted continuity for these tools
        \end{itemize}

        \textbf{In exchange, I request:}
        \begin{itemize}
            \item Access sufficient to deliver results (read-only source, bug escalation, liaison access)
            \item A formal \textbf{revenue-sharing agreement} based on the actual App Store proceeds from supported titles
            \item Governance by Apple Legal, not product teams, to avoid conflict with business-line incentives
            \item Recognition as the primary point of record for legal, compliance, and user inquiries regarding these products
        \end{itemize}

        This is a preventative posture — not a demand for restoration of a product line, but for \textbf{containment of legal exposure and user harm} Apple has not formally acknowledged.

        \section*{IV. Rationale for Legal Governance}

        This engagement must report to Apple Legal — not Developer Relations, not Product, and not an SVP or GM responsible for the original sunsetting decisions. Those individuals have already introduced risk through misaligned incentives and are unlikely to prioritize remediation.

        Legal governance ensures:
        \begin{itemize}
            \item Institutional neutrality and risk-based decision-making
            \item Alignment with Apple’s regulatory and reputational obligations
            \item Protection of user rights and restoration of platform integrity
        \end{itemize}

        I am not seeking re-employment. But should Apple wish to reinstate my original badge number under a new contractor credential, I’d consider it a symbolic return — not to the org chart, but to the mission of upholding Apple’s platform values.

        \section*{V. Response Requested}

        I request a response within 10 business days to initiate confidential discussions. This letter has not been filed in court, but it is being delivered via formal process service to ensure proper routing, timestamping, and executive-level visibility.

        I am fully prepared to proceed with regulatory complaints, press briefings, and civil filings if resolution is not pursued in good faith.

        \closing{Respectfully,\\[1em]
        \textbf{[Your Name]}\\
        \vspace{1em}
        \smallskip
        [Your Address] \\
        [Email Address] \\
        [Phone Number]}

    \end{letter}
\end{document}
